%%%%%%%%%%%%%%%%%%%%%%%%%%%%%%%%%%%%%%%%%%%%%%%%%%%%%%%%%%%%%%%%%%%%%%%%%%%
%
% Generic template for TFC/TFM/TFG/Tesis
%
% $Id: bibliography.tex,v 1.9 2015/06/05 00:10:32 macias Exp $
%
% By:
%  + Javier Macías-Guarasa. 
%    Departamento de Electrónica
%    Universidad de Alcalá
%  + Roberto Barra-Chicote. 
%    Departamento de Ingeniería Electrónica
%    Universidad Politécnica de Madrid   
% 
% Based on original sources by Roberto Barra, Manuel Ocaña, Jesús Nuevo,
% Pedro Revenga, Fernando Herranz and Noelia Hernández. Thanks a lot to
% all of them, and to the many anonymous contributors found (thanks to
% google) that provided help in setting all this up.
%
% See also the additionalContributors.txt file to check the name of
% additional contributors to this work.
%
% If you think you can add pieces of relevant/useful examples,
% improvements, please contact us at (macias@depeca.uah.es)
%
% You can freely use this template and please contribute with
% comments or suggestions!!!
%
%%%%%%%%%%%%%%%%%%%%%%%%%%%%%%%%%%%%%%%%%%%%%%%%%%%%%%%%%%%%%%%%%%%%%%%%%%%

% IMPORTANT: YOU DON'T HAVE TO EDIT THIS FILE, JUST EDIT THE bibliofiles.tex file
% IMPORTANT: YOU DON'T HAVE TO EDIT THIS FILE, JUST EDIT THE bibliofiles.tex file
% IMPORTANT: YOU DON'T HAVE TO EDIT THIS FILE, JUST EDIT THE bibliofiles.tex file
% IMPORTANT: YOU DON'T HAVE TO EDIT THIS FILE, JUST EDIT THE bibliofiles.tex file
% IMPORTANT: YOU DON'T HAVE TO EDIT THIS FILE, JUST EDIT THE bibliofiles.tex file
% IMPORTANT: YOU DON'T HAVE TO EDIT THIS FILE, JUST EDIT THE bibliofiles.tex file
% IMPORTANT: YOU DON'T HAVE TO EDIT THIS FILE, JUST EDIT THE bibliofiles.tex file
% IMPORTANT: YOU DON'T HAVE TO EDIT THIS FILE, JUST EDIT THE bibliofiles.tex file
% IMPORTANT: YOU DON'T HAVE TO EDIT THIS FILE, JUST EDIT THE bibliofiles.tex file
% IMPORTANT: YOU DON'T HAVE TO EDIT THIS FILE, JUST EDIT THE bibliofiles.tex file
% IMPORTANT: YOU DON'T HAVE TO EDIT THIS FILE, JUST EDIT THE bibliofiles.tex file
% IMPORTANT: YOU DON'T HAVE TO EDIT THIS FILE, JUST EDIT THE bibliofiles.tex file
% IMPORTANT: YOU DON'T HAVE TO EDIT THIS FILE, JUST EDIT THE bibliofiles.tex file
% IMPORTANT: YOU DON'T HAVE TO EDIT THIS FILE, JUST EDIT THE bibliofiles.tex file
% IMPORTANT: YOU DON'T HAVE TO EDIT THIS FILE, JUST EDIT THE bibliofiles.tex file
% IMPORTANT: YOU DON'T HAVE TO EDIT THIS FILE, JUST EDIT THE bibliofiles.tex file
% IMPORTANT: YOU DON'T HAVE TO EDIT THIS FILE, JUST EDIT THE bibliofiles.tex file
% IMPORTANT: YOU DON'T HAVE TO EDIT THIS FILE, JUST EDIT THE bibliofiles.tex file
% IMPORTANT: YOU DON'T HAVE TO EDIT THIS FILE, JUST EDIT THE bibliofiles.tex file
% IMPORTANT: YOU DON'T HAVE TO EDIT THIS FILE, JUST EDIT THE bibliofiles.tex file
% IMPORTANT: YOU DON'T HAVE TO EDIT THIS FILE, JUST EDIT THE bibliofiles.tex file


\ifthenelse{\equal{\bibliosystem}{biblatex}}
{
  % Use biblatex instead of bibtex

  %% Wen changing to biblatex, we could not define here the bibliography files, so that
  %% You need to edit the bibliofiles.tex file to do so

  %% % Now add all bib files to be processed
  %% \addbibresource{\myreferencespath\mybibfileOne}
  %% % \addbibresource{\myreferencespath\mybibfileTwo}
  %% % ...
  %% % \addbibresource{\myreferencespath\mybibfileN}

  \printbibliography[heading=bibintoc]


}
{
  % Use bibtex

  %\bibliographystyle{plainnat}
  %\bibliographystyle{dinat}
  %\bibliographystyle{unsrt}
  \bibliographystyle{IEEEtran}

  % The following is overly complicated because I was not able to do so in
  % another way. The problem is the bibliography command being "called"
  % from both the root and anteproyecto directories...
  %
  %% Here define as many bibfiles as needed
%%
%% It is compulsory that they are named as \mybibfileOne
%% \mybibfileTwo, \mybibfileThree, ... \mybibfileTen
%%
%% If you need more than ten, you will have to edit
%% Config/preamble.tex and Book/biblio/bibliography.tex
%% to support this adition
%%
%% The file names may change at your will, but they must
%% be in the Book/biblio directory

\newcommand{\mybibfileOne}{biblio/biblio.bib}
\newcommand{\mybibfileTwo}{biblio/nobiblio.bib}
%% \newcommand{\mybibfileThree}{AudioVisualNew.bib}
%% \newcommand{\mybibfileFour}{biblio/audiotracking.bib}
%% \newcommand{\mybibfileFive}{biblio/audiovisualtracking.bib}
%% \newcommand{\mybibfileSix}{biblio/backgroungsubstraction.bib}
%% \newcommand{\mybibfileSeven}{biblio/databases.bib}
%% \newcommand{\mybibfileEight}{biblio/evalmetrics.bib}
%% \newcommand{\mybibfileNine}{biblio/facedetect.bib}
%% \newcommand{\mybibfileTen}{biblio/facedetectADABOOST.bib}
%% \newcommand{\mybibfileEleven}{biblio/facedetectmultiview.bib}
%% \newcommand{\mybibfileTwelve}{biblio/facedetectprob2d.bib}
%% \newcommand{\mybibfileThirteen}{biblio/others.bib}
%% \newcommand{\mybibfileFourteen}{biblio/skindetect.bib}
%% \newcommand{\mybibfileFifteen}{biblio/tracking.bib}
%% \newcommand{\mybibfileSixteen}{biblio/videotracking.bib}
%% \newcommand{\mybibfileSeventeen}{biblio/voiceActivityDetection.bib}
%% \newcommand{\mybibfileEighteen}{biblio/headposeextraction.bib}
%% \newcommand{\mybibfileNineteen}{biblio/AudioVisualSpeakerTracking.bib}
%% \newcommand{\mybibfileTwenty}{biblio/BibliogPFVJ.bib}
%% \newcommand{\mybibfileTwentyone}{biblio/tools.bib}
%% \newcommand{\mybibfileTwentytwo}{biblio/infrared.bib}
%% \newcommand{\mybibfileTwentythree}{}
%% \newcommand{\mybibfileTwentyfour}{}
%% \newcommand{\mybibfileTwentyfive}{}


  \newcommand{\mybibfiles}{}
  \ifdef{\mybibfileOne}
  {
  \let\oldmybibfiles\mybibfiles
  \renewcommand{\mybibfiles}{\myreferencespath\mybibfileOne}
  }
  {
  \errorYOUmustDEFINEatLEASTmybibfileOneInbibliofilesDOTtex
  }
  \ifdef{\mybibfileTwo}
  {
  \let\oldmybibfiles\mybibfiles
  \renewcommand{\mybibfiles}{\oldmybibfiles,\myreferencespath\mybibfileTwo}
  }
  {
  }
  \ifdef{\mybibfileThree}
  {
  \let\oldmybibfiles\mybibfiles
  \renewcommand{\mybibfiles}{\oldmybibfiles,\myreferencespath\mybibfileThree}
  }
  {
  }

  \ifdef{\mybibfileFour}
  {
  \let\oldmybibfiles\mybibfiles
  \renewcommand{\mybibfiles}{\oldmybibfiles,\myreferencespath\mybibfileFour}
  }
  {
  }
  \ifdef{\mybibfileSix}
  {
  \let\oldmybibfiles\mybibfiles
  \renewcommand{\mybibfiles}{\oldmybibfiles,\myreferencespath\mybibfileSix}
  }
  {
  }

  \ifdef{\mybibfileSeven}
  {
  \let\oldmybibfiles\mybibfiles
  \renewcommand{\mybibfiles}{\oldmybibfiles,\myreferencespath\mybibfileSeven}
  }
  {
  }

  \ifdef{\mybibfileEight}
  {
  \let\oldmybibfiles\mybibfiles
  \renewcommand{\mybibfiles}{\oldmybibfiles,\myreferencespath\mybibfileEight}
  }
  {
  }

  \ifdef{\mybibfileNine}
  {
  \let\oldmybibfiles\mybibfiles
  \renewcommand{\mybibfiles}{\oldmybibfiles,\myreferencespath\mybibfileNine}
  }
  {
  }

  \ifdef{\mybibfileTen}
  {
  \let\oldmybibfiles\mybibfiles
  \renewcommand{\mybibfiles}{\oldmybibfiles,\myreferencespath\mybibfileTen}
  }
  {
  }

  \ifdef{\mybibfileEleven}
  {
    \let\oldmybibfiles\mybibfiles
    \renewcommand{\mybibfiles}{\oldmybibfiles,\myreferencespath\mybibfileEleven}
  }
  {
  }

  \ifdef{\mybibfileTwelve}
  {
    \let\oldmybibfiles\mybibfiles
    \renewcommand{\mybibfiles}{\oldmybibfiles,\myreferencespath\mybibfileTwelve}
  }
  {
  }

  \ifdef{\mybibfileThirteen}
  {
    \let\oldmybibfiles\mybibfiles
    \renewcommand{\mybibfiles}{\oldmybibfiles,\myreferencespath\mybibfileThirteen}
  }
  {
  }

  \ifdef{\mybibfileFourteen}
  {
    \let\oldmybibfiles\mybibfiles
    \renewcommand{\mybibfiles}{\oldmybibfiles,\myreferencespath\mybibfileFourteen}
  }
  {
  }

  \ifdef{\mybibfileFifteen}
  {
    \let\oldmybibfiles\mybibfiles
    \renewcommand{\mybibfiles}{\oldmybibfiles,\myreferencespath\mybibfileFifteen}
  }
  {
  }

  \ifdef{\mybibfileSixteen}
  {
    \let\oldmybibfiles\mybibfiles
    \renewcommand{\mybibfiles}{\oldmybibfiles,\myreferencespath\mybibfileSixteen}
  }
  {
  }

  \ifdef{\mybibfileSeventeen}
  {
    \let\oldmybibfiles\mybibfiles
    \renewcommand{\mybibfiles}{\oldmybibfiles,\myreferencespath\mybibfileSeventeen}
  }
  {
  }

  \ifdef{\mybibfileEighteen}
  {
    \let\oldmybibfiles\mybibfiles
    \renewcommand{\mybibfiles}{\oldmybibfiles,\myreferencespath\mybibfileEighteen}
  }
  {
  }

  \ifdef{\mybibfileNineteen}
  {
    \let\oldmybibfiles\mybibfiles
    \renewcommand{\mybibfiles}{\oldmybibfiles,\myreferencespath\mybibfileNineteen}
  }
  {
  }

  \ifdef{\mybibfileTwenty}
  {
    \let\oldmybibfiles\mybibfiles
    \renewcommand{\mybibfiles}{\oldmybibfiles,\myreferencespath\mybibfileTwenty}
  }
  {
  }

  \ifdef{\mybibfileTwentyone}
  {
    \let\oldmybibfiles\mybibfiles
    \renewcommand{\mybibfiles}{\oldmybibfiles,\myreferencespath\mybibfileTwentyone}
  }
  {
  }

  \ifdef{\mybibfileTwentytwo}
  {
    \let\oldmybibfiles\mybibfiles
    \renewcommand{\mybibfiles}{\oldmybibfiles,\myreferencespath\mybibfileTwentytwo}
  }
  {
  }

  \ifdef{\mybibfileTwentythree}
  {
    \let\oldmybibfiles\mybibfiles
    \renewcommand{\mybibfiles}{\oldmybibfiles,\myreferencespath\mybibfileTwentythree}
  }
  {
  }

  \ifdef{\mybibfileTwentyfour}
  {
    \let\oldmybibfiles\mybibfiles
    \renewcommand{\mybibfiles}{\oldmybibfiles,\myreferencespath\mybibfileTwentyfour}
  }
  {
  }

  \ifdef{\mybibfileTwentyfive}
  {
    \let\oldmybibfiles\mybibfiles
    \renewcommand{\mybibfiles}{\oldmybibfiles,\myreferencespath\mybibfileTwentyfive}
  }
  {
  }

  % Do not touch this
  % The commands around the \bibliography{} command should be included if using bibtex, but
  % according to Gonzalo Corral's PR on July 2022, they break compilation in overleaf. I think
  % this should not happen if the bib files are in iso-8859-1 when using bibtex instead of
  % biber, but need to be tested (TODO)
  \inputencoding{latin1}
  \bibliography{\mybibfiles}
  \inputencoding{utf8}
}

%%% Local Variables:
%%% TeX-master: "../book"
%%% coding: utf-8
%%% End:


