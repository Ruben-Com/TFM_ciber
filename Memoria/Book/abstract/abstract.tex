%%%%%%%%%%%%%%%%%%%%%%%%%%%%%%%%%%%%%%%%%%%%%%%%%%%%%%%%%%%%%%%%%%%%%%%%%%%
%
% Generic template for TFC/TFM/TFG/Tesis
%
% $Id: abstract.tex,v 1.9 2015/06/05 00:10:31 macias Exp $
%
% By:
%  + Javier Macías-Guarasa. 
%    Departamento de Electrónica
%    Universidad de Alcalá
%  + Roberto Barra-Chicote. 
%    Departamento de Ingeniería Electrónica
%    Universidad Politécnica de Madrid   
% 
% Based on original sources by Roberto Barra, Manuel Ocaña, Jesús Nuevo,
% Pedro Revenga, Fernando Herránz and Noelia Hernández. Thanks a lot to
% all of them, and to the many anonymous contributors found (thanks to
% google) that provided help in setting all this up.
%
% See also the additionalContributors.txt file to check the name of
% additional contributors to this work.
%
% If you think you can add pieces of relevant/useful examples,
% improvements, please contact us at (macias@depeca.uah.es)
%
% You can freely use this template and please contribute with
% comments or suggestions!!!
%
%%%%%%%%%%%%%%%%%%%%%%%%%%%%%%%%%%%%%%%%%%%%%%%%%%%%%%%%%%%%%%%%%%%%%%%%%%%

\chapter*{Abstract}\label{cha:abstract}

\addcontentsline{toc}{chapter}{Abstract}

%En este TFM se ha llevado a cabo la adaptación y ejecución de varios algoritmos criptográficos postcuánticos, entre los cuales se encuentran Dilithium, Kyber y Sphincs en distintos dispositivos relativos a IoT, como son ESP32, RP2040 y STM32.

%Posteriormente, se ha llevado a cabo la medición de rendimiento de estas implementaciones y dichas mediciones han sido comparadas.

In this TFM, multiple cryptographic post-quantum algorithms, such as Dilithium, Kyber and Sphincs, have been adapted and executed in various platforms relative to IoT, like ESP32, RP2040 and STM32.

Afterwards, these implementations' performance have been measured and, finally, these measurements have been compared.

\textbf{Keywords:} \myThesisKeywordsEnglish.

%%% Local Variables:
%%% TeX-master: "../book"
%%% End:


