%%%%%%%%%%%%%%%%%%%%%%%%%%%%%%%%%%%%%%%%%%%%%%%%%%%%%%%%%%%%%%%%%%%%%%%%%%% 
% 
% Generic template for TFC/TFM/TFG/Tesis
% 
% By:
%  + Javier Macías-Guarasa.
%    Departamento de Electrónica
%    Universidad de Alcalá
%  + Roberto Barra-Chicote.
%    Departamento de Ingeniería Electrónica
%    Universidad Politécnica de Madrid
% 
% By: + Javier Macías-Guarasa. Departamento de Electrónica Universidad de Alcalá + Roberto Barra-Chicote. Departamento de Ingeniería Electrónica Universidad Politécnica de Madrid
% 
% Based on original sources by Roberto Barra, Manuel Ocaña, Jesús Nuevo, Pedro Revenga, Fernando Herránz and Noelia Hernández. Thanks a lot to all of them, and to the many anonymous contributors found (thanks to google) that provided help in setting all this up.
% 
% See also the additionalContributors.txt file to check the name of additional contributors to this work.
% 
% If you think you can add pieces of relevant/useful examples, improvements, please contact us at (macias@depeca.uah.es)
% 
% You can freely use this template and please contribute with comments or suggestions!!!
% 
%%%%%%%%%%%%%%%%%%%%%%%%%%%%%%%%%%%%%%%%%%%%%%%%%%%%%%%%%%%%%%%%%%%%%%%%%%% 

\chapter{Introducción}\label{cha:introduccion}

En este capítulo se explica la motivación y necesidad que han llevado a la ejecución de este trabajo, así como los distintos objetivos que se busca conseguir a lo largo de su realización.
Adicionalmente, se especifica la estructura por la que se rige este documento.


\section{Objetivos del proyecto}\label{sec:objetivos}

El objetivo final de este trabajo consiste en la implementación de algoritmos criptográficos postcuánticos en distintos dispositivos \ac{IoT}.
Para poder alcanzar este objetivo, se han planteado una serie de pautas cuyo seguimiento conduce al objetivo final:

\begin{itemize}
    \item \textbf{Selección de algoritmos a implementar}: En primer lugar, se deberá elegir que algoritmos se buscará implementar.
    \item \textbf{Selección de plataformas \ac{IoT} a emplear}: A continuación, se llevará a cabo una selección de los dispositivos \ac{IoT} en los que se implementarán los algoritmos previamente mencionados.
    \item \textbf{Análisis de compatibilidad}: El siguiente objetivo consistirá en llevar a cabo un análisis de compatibilidad existente entre los algoritmos y los dispositivos seleccionados.
    \item \textbf{Validación, diseño de pruebas y análisis de resultados}: Finalmente, se comprobará el funcionamiento de los algoritmos mediante pruebas y el posterior análisis de los resultados.
\end{itemize}


\section{Estructura de la memoria}\label{sec:estructura}

Después de haber explicado la motivación el proyecto así como los objetivos que se buscan mediante el mismo, se indica la estructura que seguirá este documento:

\begin{enumerate}
    \item \textbf{Introducción}: En este capítulo se realiza una contextualización del proyecto. Además, se indican los objetivos buscados en dicho trabajo y la estructura que sigue este documento explicativo.
    \item \textbf{Estado del arte}: A lo largo de este capítulo se detallan una serie de conocimientos previos que el lector debe conocer para poder comprender correctamente el funcionamiento y desarrollo del proyecto.
    \item \textbf{Desarrollo}: En este capítulo se presentan las vías de trabajo llevadas a cabo para conseguir los objetivos mencionados en el capítulo introductorio.
    \item \textbf{Resultados}: A lo largo de este capítulo se muestran las diferentes pruebas realizadas a los algoritmos en los diferentes dispositivos utilizados.
    \item \textbf{Conclusiones y trabajo futuro}: Se finaliza esta memoria con un capítulo que comente las conclusiones extraíbles de los resultados explicados en el apartado anterior. También, se indicarán posibles vías de desarrollo para mejorar la implementación final y añadir características a la misma.
    \item \textbf{Bibliografía}: Se incluyen todos los documentos consultados para la realización de este trabajo, ya sean páginas web, libros o artículos. Dichos documentos estarán citados siguiendo el estilo indicado por el \ac{IEEE}.
    \item \textbf{Anexo A}: Finalmente, se añade un anexo en el cual se indica la planificación temporal seguida durante la ejecución de este proyecto y el coste de la realización del mismo.
\end{enumerate}



%%% Local Variables:
%%% TeX-master: "../book"
%%% End:


